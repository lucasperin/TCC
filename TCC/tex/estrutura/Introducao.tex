\chapter{Introdução}
O uso de sistemas de bancos de dados tornou-se recorrente para os mais diversos tipos de aplicações. Com o uso extensivo da internet, as
aplicações que utilizam sistemas de bancos de dados \textit{online} são cada vez mais comuns.

Aplicações que utilizam bancos de dados \textit{online} normalmente armazenam dados sensíveis, como salários e outras informaçõees pessoais \cite{Kamel.integrity.2009}.
Devido ao conteúdo potencialmente sigiloso, o acesso ou a modificação não autorizada a tais dados pode não ser desejado.
Dessa forma, o sigilo e integridade de sistemas de banco de dados tem atraído pesquisadores das áreas de banco de dados e segurança.
Os sistemas de banco de dados, quando utilizados em ambientes compartilhados, possuem diversas ameaças de segurança provenientes de
usuários não-autorizados, mau-uso e ameaças externas. Baseando-se nestas características, pode-se classificar as ameaças em quatro
tipos principais:
\begin{enumerate}
 \item leitura não autorizada de dados;
 \item modificação não autorizada de dados;
 \item remoção não autorizada de dados;
 \item adição não autorizada de dados.
\end{enumerate}
%\TODO{Está OK para TCC1, adicionar mais dados para a monografia. Dados com datas, marcos históricos etc..}

\section{Objetivos}

\subsection{Gerais}
Este trabalho tem como objetivo geral apresentar soluções plausíveis para garantir a integridade de bancos de dados, protegendo o conteúdo sensível 
de ataques maliciosos ou de modificações não autorizadas. Além disso, deverá ser possível detectar alterações indevídas n

\subsection{Específicos}
\begin{itemize}
  \item Elaborar uma biblioteca em Java de com a finalidade de prover suporte para as soluções apresentadas neste trabalho;
  \item Providenciar suporte aos dispositivos criptográficos existentes no mercado com a biblioteca criada, incrementando novas camadas de 
  segurança para as aplicações que fizerem uso da biblioteca;
  \item Facilitar o desenvolvimento de aplicações de \textit{software} com garantia de integridade de bancos de dados;
  \item Manter as soluções apresentadas dentro de uma faixa de desempenho atraente para aplicações que possam vir a utilizar a biblioteca criada.
\end{itemize}

\section{Justificativa}
O problema da leitura não autorizada de dados (sigilo dos dados) já foi pesquisado extensivamente e normalmente é resolvido através da
cifração do banco de dados \cite{samarati.encryption.2006, Samarati:2010}, em conjunto com métodos de indexação
\cite{samarati.encryption.2006, samarati.indexing.2003}, para agilizar as consultas.
Entretanto, os problemas de modificações não autorizadas de dados (integridade dos dados) ainda necessitam de soluções eficientes
\cite{Samarati:2010}.
Os métodos existentes para tal problema geralmente requerem o desenvolvimento de um novo \ac{SGBD} ou
alterações significantes nos \ac{SGBD}s já existentes \cite{Xie.integrity.2007}. Adicionalmente, soluções disponíveis no mercado possuem 
custo elevado de implantação e manutenção, inviáveis para empresas de pequeno e médio porte que também necessitam de tais serviços de segurança.

\section{Metodologia}
\TODO{Trocar o tempo verbal? Ex: ``Como ela está contruida e como seus serviços são acessados'' por ``Como ela SERÁ contruida e como seus serviços SERÃO..''}

Neste trabalho, será estudada toda a fundamentação teórica necessária para compreender as soluções propostas. Será demonstrado como 
as ameaças citadas anteriormente podem ser combatidas utilizando \ac{HMAC} e a cifração de dados sigilosos. Em seguida, serão comentadas 
e justificadas as tecnologias utilizadas para o desenvolvimento da biblioteca. Em posse desse conhecimento, pode-se finalmente elaborar mais sobre a solução.
O próximo passo, será demonstração da fundamentação da biblioteca. Como ela está construida e como que seus serviços são acessados pelas aplicações. É importante 
levantar que uma biblioteca de \textit{software} não constitui uma aplicação por si só. Ela apenas contribui com os recursos necessários para soluções.
Em um último momento, as soluções criadas neste trabalho serão testadas em divérsos cenários. Dessa forma, pode-se demonstrar o custo de desempenho para a 
utilização da biblioteca de integridade de bancos de dados.

%\section{Limitações do Trabalho}
%\TODO{Verificar a necessidade. Usar TCC Dexter como exemplo}

