% resumo
\begin{resumo}
Modifica\c{c}\~{o}es n\~{a}o autorizadas a sistemas de banco de dados podem causar preju\'{i}zos para pessoas e organiza\c{c}\~{o}es, sendo
de extrema import\^{a}ncia a garantia de sigilo e integridade a tais sistemas. Geralmente aplica\c{c}\~{o}es utilizam os recursos
disponibilizados por \ac{SGBD} para assegurar o sigilo e a integridade dos dados armazenados
no \ac{SGBD}. Entretanto, os \ac{SGBD}s que prov\^{e}m os recursos necess\'{a}rios para garantir o sigilo e a integridade dos dados possuem um custo muito elevado,
muitas vezes invi\'{a}vel para organiza\c{c}\~{o}es de m\'{e}dio e pequeno porte. Este trabalho analisa e implementa um m\'{e}todo de verifica\c{c}\~{a}o de integridade de dados, baseado no uso de \ac{HMAC},
independente de \ac{SGBD}, aproveitando estruturas possívelmente disponíveis como \ac{TPM} e {\it smart cards} entre outros dispositivos criptográficos.
Os testes realizados mostram a efici\^{e}ncia do m\'{e}todo. Em m\'{e}dia, a sobrecarga de processamento para o c\'{a}lculo e verifica\c{c}\~{a}o do \ac{HMAC} para um
registro do banco de dados n\~{a}o ultrapassa 100\% do tempo de execu\c{c}\~{a}o de determinada opera\c{c}\~{a}o.

\TODO{não precisa das palavras chaves?}
\end{resumo}

