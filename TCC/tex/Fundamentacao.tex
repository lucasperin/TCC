\chapter{Fundamentação Teórica}

\section{Criptografia}
\textit{Criptografia} é o estudo e a arte de transformar informação de forma que fique ilégivel ou incompreensível para todos exceto 
a quem ela for destinada. Utilizada quase que exclusivamente nos seus dias mais atuais em circunstâncias diplomaticas e militares, hoje 
é mais do que um simples meio de trocar informação secreta \cite{Luciano}. Existe a necessidade urgente de proteger a vasta quantidade de dados existêntes 
em meios digitais, mesmo aqueles em ambientes supostamente seguros. 
De acordo com Housley, criptografia pode ser definida como:
%\TODO{Continuar explorando o artigo do Luciano para relatar mais sobre história da criptografia (TCC2)}
\begin{quote}
\footnotesize{
A palavra criptografia significa escondido ou escrita secreta. Criptografia é conhecida geralmente como o 
embaralhamento e o desembaralhamento de mensagens privadas. Uma mensagem é embaralhada para manter sua privacidade 
ou para proteger sua confidencialidade. Técnicas modernas de criptografia são também usadas para determinar caso uma mensagem 
foi alterada após o momento de ter sido criada e para identificar a origem da mesma. Uma mensagem não adulterada tem integridade. 
O conhecimento da origem de uma mensagem significa autenticidade.
} \cite[p.-5]{Planning}
\end{quote}

Existem diversos algoritmos criptográficos para cifragem e decifragem de mensagens. Em sua maioria, faz-se uso de duas entradas para gerar uma mensagem cifrada. 
Uma das entrada é o conteudo da informação sigilosa e a outra é um valor secreto denominado \textit{chave}. Depedendo se o algorítmo é dito simétrico ou asimétrico, 
poderão uma ou mais chaves funcionando de maneiras diferentes. 

\subsection{Criptografia Simétrica}
Em \textit{cripotografia simétrica}, usa-se apenas uma chave secreta entre quem envia e quem recebe a mensagem cifrada. É por este motivo que sistemas que fazem uso de 
criptografia simétrica podem também ser chamados de sistemas de \textit{chaves secretas compartilhadas} \cite{Planning}. Em um cenário onde deseja-se enviar uma mensagem 
cifrada para alguém, deve-se garantir que essa chave seja uma segredo entre destino e destinatário. A mesma chave será utilizada no processo de cifragem por quem envia e também no 
processo de decifragem por quem recebe.
   
\subsection{Criptografia Assimétrica}
Enquanto o uso de chaves simétricas pode ser bastante conveniênte, o gerenciamento das mesmas pode se tornar bastante complexo. Em alguns casos, 
deseja-se manter uma vasta quantidade de chaves para comunicação segura entre diversos periféricos. Nesses casos algoritmos de comunicação, por exemplo, podem se tornar pouco eficiêntes devido a 
necessidade de gerenciar uma chave para cada destino diferente.

\textit{Criptografia assimétrica} ou \textit{criptografia de chaves públicas}, utiliza duas chaves distintas no lugar de uma. Uma delas chamada de \textit{chave privada} e a outra chamada de \textit{chave pública}. 
Ambas chaves são complementares, porém, nunca deve ser possível obter-se a chave privada através da chave pública \cite{Planning}. O uso de chaves públicas simplifica bastante o processo de gerenciamente de chaves 
pelo fato de reduzir drasticamente o número de chaves que serão gerenciadas. Por outro lado, algorítmos de cifração de dados através do uso de chaves públicas não são tão eficiêntes quando os de chave simétrica. 
Adicionalmente, pelo fato de existir uma chave pública capaz de decifrar a informação protegida, esta informação não possui mais confidencialidade, apenas autenticidade.

Existem diversos usos para criptografia de chaves públicas. Entre eles, a troca de chaves públicas para gerar uma chave simétrica conhecida entre apenas dois periféricos \cite{KeyAgreement}. Outro exemplo é 
a \ac{ICP}, mantendo uma cadeia de entidades com posse de chaves privadas, é capaz de garantir autenticidade de documentos eletrônicos para usuários \cite{Planning}. 

\subsection{Resumo criptográfico}
\TODO {precisa disso? usamos o hmac, talvez só falar dele}
\TODO{Chave de seção?}

\section{Bancos de Dados}

\section{Tecnologias usadas}
\TODO{Lucas Martins aconselhou não utilizar as subsections aqui. Resumir todo o conteudo em alguns parágrafos. Talvez cortar algumas tecnologias como Java, xml, Junit, etc..}

\subsection{Bouncy Castle}

\subsection{LibCryptodev}

\subsection{Java}

\subsection{SQLite}

\subsection{Subversion}

\subsection{JUnit}

\subsection{XML}

\subsection{TPMJ}
\TODO{foi usado o tpmj porque o modulo pkcs\#11 para o acesso ao tpm era bastante instável}



\TODO{foi usado mais alguma coisa, mais alguma coisa precisa ser colocada aqui?}

\TODO{Keystore, vai na criptografia?}

\TODO{O PKCS\#11 vai nisso?, e PKCS\#12}

\TODO{falar o que é um provider?}

\TODO{chaves PEM e DER, falar?, deu problemas com isso no TPMJ}

\TODO{Ant? Foi usado isso para fazer os builds das versões}
