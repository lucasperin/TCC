\chapter{Integridade e Sigilo}

\TODO{Já está bem comentado na intro. Trocar por uma breve introdução do capitulo.}

Um dos problemas mais comuns quando se trata do sigilo em base de dados \'{e} a leitura n\~{a}o autorizada de informa\c{c}\~{o}es sens\'{i}veis presentes nas tabelas.
A prote\c{c}\~{a}o contra este tipo de acesso pode ser ser feita a n\'{i}vel de aplica\c{c}\~{a}o, o que \'{e} de fato realizada na maioria dos casos.
Para tal, sugere-se a cifração de dados em colunas das tabelas.
Deve-se selecionar colunas onde existem informa\c{c}\~{o}es sigilosas nas tabelas e mante-las cifradas. A cifração destes dados pode ser feita diretamente pela aplica\c{c}\~{a}o
ou pelo \ac{SGBD}.

Para não afetar muito o desempenho das consultas, é comum o uso de métodos de indexação juntamente com a cifração dos dados \cite{samarati.encryption.2006, samarati.indexing.2003, hakan.indexing.2004}.

Para evitar a modificação não autorizada de registros contidos na base de dados, a aplicação deve restringir e definir as permissões de cada usuário.
O \ac{SGBD} também deve conter um mapeamento correto de usuários e seus respectivos privilégios a fim de impedir a execução de ações não autorizadas.
Entretanto, existem perfis de usuários, como os administradores do servidor, \ac{DBA} e programadores, que podem acessar o \ac{SGBD} sem o conhecimento da aplicação,
podendo ocorrer situações onde, embora o usuário não seja autorizado a realizar determinada modificação, ele tenha permissões suficientes no sistema que permitem que ele o faça, mesmo que não intencionalmente.
A cifração de colunas que contém dados sensíveis já reduz consideravelmente este problema, uma vez que um usuário mal-intencionado não conseguirá modificar o registro por não possuir a chave,
 e possivelmente sequer será capaz de identificar qual registro ele deve modificar para obter os efeitos que ele deseja. Embora as chances do atacante sejam reduzidas, ainda há a possibilidade de realizar modificação de registros 
 a fim de corromper o sistema.

Infelizmente, os \ac{SGBD}s mais comuns, como MySQL, PostreSQL, Firebird, etc, n\~{a}o possuem mecanismos para evitar tais problemas.
Outros sistemas mais espec\'{i}ficos possuem op\c{c}\~{o}es mais avan\c{c}adas, por\'{e}m também possuem custo bastante elevado.
Para prover tal funcionalidade, foi estudada uma proposta que faz uso de \ac{HMAC} \cite{Bellare:1996:KHF:646761.706031, rfc6151}.

\section{Adi\c{c}\~{a}o do c\'{a}lculo do \ac{HMAC} nos registros da base de dados}

Para se implementar este mecanismo, a aplica\c{c}\~{a}o ser\'{a} a \'{u}nica respons\'{a}vel pela manipula\c{c}\~{a}o dos registros da base de dados, e, para prover tal garantia,
a aplica\c{c}\~{a}o ter\'{a} uma chave sim\'{e}trica conhecida somente por ela. Esta chave, por sua vez, ser\'{a} utilizada para gerar o \ac{HMAC} dos registros.

Atrav\'{e}s da utiliza\c{c}\~{a}o do \ac{HMAC}, \'{e} poss\'{i}vel detectar qualquer modifica\c{c}\~{a}o não autorizada realizada nos registros. Isto deve-se ao fato de que o atacante não tem conhecimento
da chave necessária para gerar o \ac{HMAC}, impossibilitando-o que consiga calcular um valor de \ac{HMAC} válido. Este mecanismo tamb\'{e}m soluciona a quest\~{a}o de adi\c{c}\~{a}o de registros pelo atacante,
pois novamente, este necessitar\'{a} da chave da aplica\c{c}\~{a}o para realizar o cálculo do \ac{HMAC} de forma correta.

\section{Exemplo de implementa\c{c}\~{a}o}

Para exemplificar o funcionamento do método utilizando \ac{HMAC}, considere uma tabela de pessoas, com as colunas id, nome, email e a coluna para armazenar o valor do \ac{HMAC}. Os dados s\~{a}o apresentados na Tabela \ref{tab:exemplo}.

\begin{table}[htb]\footnotesize
  \begin{center}
    \caption{Tabela de exemplo de dados.}\label{tab:exemplo}
    \begin{tabular}{|l|l|l|l|}\hline
      \textbf{\underline{id}} & \textbf{nome} & \textbf{email} & \textbf{hmac} \\ \hline
       1 & Joao & joao@foo.com.br & abcd \\ \hline
      2 & Maria & maria@foo.com.br & qwer \\ \hline
      3 & Ana & ana@foo.com.br & kjhd \\ \hline
      4 & Roberto & roberto@foo.com.br & vpiu \\ \hline
    \end{tabular}
  \end{center}
\end{table}

\subsection{Adicionando registros}

Para adicionar um novo registro, a aplicação deve primeiramente calcular o valor do \ac{HMAC} para o novo registro. Esse cálculo é feito através da concatenação dos valores de todas as colunas da tabela.
Para adicionar uma nova pessoa com o nome ``Jose'' e email ``jose@foo.com.br'', o cálculo do \ac{HMAC} é apresentado na expressão (\ref{eq:hmac-exemplo}).

\begin{equation} \label{eq:hmac-exemplo}
    HMAC(chave, Josejose@foo.com.br) = ohn4
\end{equation}

Ap\'{o}s o calculo do \ac{HMAC}, a aplica\c{c}\~{a}o envia o comando de insert ao banco de dados, conforme ilustrado no trecho de código \ref{fonte:exemplo-insert}.

\lstset{language=SQL,
	basicstyle=\small,
        breaklines=true,
        numbersep=5pt,
        xleftmargin=.35in,
        xrightmargin=.35in}
\begin{center}
\begin{lstlisting}[label=fonte:exemplo-insert, caption=Código SQL para inserir um registro com HMAC]
INSERT INTO exemplo (nome, email, hmac)
VALUES ('Jose',
        'jose@foo.com.br',
        'ohn4');
\end{lstlisting}
\end{center}

\subsection{Modificando registros}

A modificação de um registro é semelhante à adição. Por exemplo, para a alteração do registro número 3, da Tabela \ref{tab:exemplo},
primeiramente deve-se consultar o banco de dados para obter o registro, conforme ilustrado no trecho de código \ref{fonte:exemplo-select}.
O pr\'{o}ximo passo \'{e} calcular o \ac{HMAC}, apresentado na express\~{a}o (\ref{eq:hmac-exemplo-2}).
Por fim, atualiza-se o registro com o comando update ilustrado no trecho de código \ref{fonte:exemplo-update}.

\lstset{language=SQL,
	basicstyle=\small,
        breaklines=true,
        numbersep=5pt,
        xleftmargin=.35in,
        xrightmargin=.35in}
\begin{center}
\begin{lstlisting}[label=fonte:exemplo-select, caption=Código SQL para selecionar um registro]
    SELECT nome, email FROM exemplo WHERE id='3';
\end{lstlisting}
\end{center}

\begin{equation} \label{eq:hmac-exemplo-2}
    HMAC(chave, Anaana.new@foo.com.br) = m3cx
\end{equation}

\lstset{language=SQL,
	basicstyle=\small,
        breaklines=true,
        numbersep=5pt,
        xleftmargin=.35in,
        xrightmargin=.35in}
\begin{center}
\begin{lstlisting}[label=fonte:exemplo-update, caption=Código SQL para atualizar um registro com HMAC]
    UPDATE exemplo SET email='ana.new@foo.com.br', hmac='m3cx'
    	WHERE id='3';
\end{lstlisting}
\end{center}

\subsection{Verificando a integridade de registros}

Por fim, para verifica\c{c}\~{a}o da integridade de um registro, deve ser feita uma consulta ao banco de dados pelo registro. Calcula-se o \ac{HMAC} para o registro e compara o \ac{HMAC} calculado com o \ac{HMAC} obtido do banco de dados.
A igualdade desses valores indica que o registro está íntegro.

\section{Medidas de prote\c{c}\~{a}o contra remo\c{c}\~{a}o n\~{a}o autorizada}

Atrav\'{e}s da utiliza\c{c}\~{a}o do \ac{HMAC}, j\'{a} \'{e} poss\'{i}vel detectar qualquer modifica\c{c}\~{a}o e adi\c{c}\~{a}o não autorizada de registros.
Entretanto, não é possível detectar a remoção não autorizada de registros.

Para o problema de remoção não autorizada de registros, sugere-se a criação de uma nova coluna para as tabelas do banco de dados, chamada de ``Histórico cifrado''.
O hist\'{o}rico cifrado possui as seguintes caracter\'{i}sticas:

\begin{itemize}
	\item permite relacionar dois ou mais registros de forma que possa se detectar a aus\^{e}ncia de um deles, caso este seja removido;
	\item n\~{a}o permitir que uma terceira parte possa calcular o ``hist\'{o}rico cifrado'' sem conhecer as chaves de cifração;
	\item utiliza\c{c}\~{a}o de opera\c{c}\~{o}es de baixo custo computacional: criptografia sim\'{e}trica e a opera\c{c}\~{a}o l\'{o}gica ``ou exclusivo'' (XOR);
	\item requer pouco espa\c{c}o de armazenamento;
	\item n\~{a}o permite que sejam detectadas dele\c{c}\~{o}es dos $n$ \'{u}ltimos registros.
\end{itemize}

Para calcular o hist\'{o}rico cifrado de um registro $n$, obtém-se o \ac{HMAC} desse registro e do registro anterior a ele. Em seguida é aplicada a função XOR nestes \ac{HMAC}s e o resultado é cifrado
com uma chave simétrica. Esse cálculo é apresentado na expressão (\ref{eq:historico}).

\begin{equation} \label{eq:historico}
\small
HistoricoCifrado(chave, HMAC_{n}, HMAC_{n-1}) = Cifração(k, (HMAC_{n} \oplus HMAC_{n-1}))
\end{equation}

\subsection{Removendo um registro}

Para remover um registro $n$, deve-se excluir o registro e atualizar o hist\'{o}rico cifrado do registro $n+1$. Para atualizar o hist\'{o}rico cifrado, utiliza-se o \ac{HMAC} registro $n+1$ e $n-1$.

\subsection{Verificando se um registro foi removido}

Para verificar se um registro $t_{n}$, de uma tabela $T$ foi removido, a seguinte propriedade deve ser satisfeita:
\begin{equation} \label{eq:validacaoHistorico}
\forall t_{n}, t_{n+1} \in T : t_{n}.HistoricoCifrado = HistoricoCifrado (chave, t_{n}, t_{n+1}) 
\end{equation}
\subsection{Remoç\~{a}o do ultimo registro}
Esta proposta possui uma vulnerabilidade, o método apresentado não detecta quando os últimos registros são removidos indevidamente, uma vez que a verificação é feita com base no registo anterior.
 Uma possível solução para o problema consiste em adicionar ao final da tabela uma enupla com valores aleatórios conhecidos. 
 Dessa forma, se o último registro for removido, poderá ser identificado, uma vez que os valores de controle não estarão mais presentes.
 \TODO{Desenvolver melhor a ideia explicando como serão afetadas as operações CRUD. Mencionar também solução circular com citação do artigo do Anderson}
 
 
 \TODO{Começar explicando a ideia? E depois o que foi feito para ela existir? Tem que falar da biblioteca, dos provideres para os cada dispositivo
 (e do callback remoto?), do gerenciador de chaves pkcs\#12, e outros subprojetos}
